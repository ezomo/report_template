\documentclass{jreport}  % カスタムクラスを使用
\begin{document}

\begin{titlepage}
\headlinefont
    \begin{center}
        \vspace*{2cm}
        {\fontsize{18pt}{20pt}\selectfont J3\quad 情報システム実験実習\quad 実験報告書} \\[2.2cm]
        {\fontsize{20pt}{20pt}\selectfont 題目\quad \underline{     ダイオードの特性測定     }} \\[3cm]
        \fontsize{14pt}{20pt}\selectfont \begin{tabular}{l l}
            実施年月日 \underline{2025年 5月 13日}& \\[0.5cm]
            提出年月日 \underline{2025年 6月 6日}& \\[4.0cm]
            提出者 & \\[0.5cm]
            通し番号 \underline{27} 学籍番号 \underline{23094} 名前 \underline{杉村実紀} \\
        \end{tabular}
        \vfill
    \end{center}
\end{titlepage}

\section{目的}
直流安定化電源,オシレータおよびデジタル・オシロスコープにより観察した直流・交流信号を基に解析し,代表的な非線形素子の一つであるダイオードの基本的な特性をデジタル・オシロスコープの波形観測を基にして理解する.

\section{実験方法}
Webclass上の資料,及びRaspberry Piを用いて,実験を進める.実際に教員の指示に従いコマンドを実行しコマンドの意味を理解する.

\section{実験結果}
\begin{enumerate}[(1). ]
    \item lsコマンド及び -a , -l オプションを用いて,ルートディレクトリ直下の隠しファイルを含むファイルとディレクトリの一覧を,詳細を含めて表示した画面を図1に示す.
          \addfigure{./pictures/1.png}{ルートディレクトリ直下の隠しファイルを含むファイルとディレクトリの一覧を,詳細を含めて表示した画面}
    \item	ワイルドカード「*」を含むコマンドを実行した画面を図2に示す
          \addfigure{./pictures/2.png}{ワイルドカード「*」を含むコマンドを実行した画面}
    \item mkdir,cp,mvコマンドを用いて,ホームディレクトリ直下に新しいディレクトリを作成し,このディレクトリ内で新しいファイルの作成,ファイルのコピー作成,ファイル名の変更を行った画面を図3に示す.
          \addfigure{./pictures/3.png}{ホームディレクトリ直下に新しいディレクトリを作成し,このディレクトリ内で新しいファイルの作成,ファイルのコピー作成,ファイル名の変更を行った画面}
    \item inコマンドおよび,-s オプションを用いて,既存のファイルとディレクトリそれぞれに対するシンボリックリンクを作成した画面を図4に示す.
          \addfigure{./pictures/4.png}{既存のファイルとディレクトリそれぞれに対するシンボリックリンクを作成し,lsコマンドで表示した画面}
    \item 自作したファイルに対して,chmodコマンドのシンボルモードにより,グループとその他のユーザにのみ,書き込みと実行の権限のみがあるように設定し,ls -l コマンドでファイルモードを表示した画面を図5に示す.
          \addfigure{./pictures/5.png}{ 自作したファイルに対して,chmodコマンドのシンボルモードにより,グループとその他のユーザにのみ,書き込みと実行の権限のみがあるように設定し,ls -lコマンドでファイルモードを表示した画面}
    \item 自作したディレクトリに対して,chmod コマンドの数値モードにより,オーナーとその他のユーザにのみ,読み書きの権限のみがあるように設定し,ls -l コマンドでファイルモードを表示した画面を図6に示す.
          \addfigure{./pictures/6.png}{自作したディレクトリに対して,chmod コマンドの数値モードにより,オーナーとその他のユーザにのみ,読み書きの権限のみがあるように設定し,ls -lコマンドでファイルモードを表示した画面}
\end{enumerate}

\section{調査}
「「rm -rf /」というコマンドの危険性について,コマンドの意味順に説明し,その危険性を解説する.
まず初めに,「rm」はremoveの略で,ファイルを削除するコマンドである[1].
次に,「-rf」とは,--forceと--recursiveの2つの短縮形オプションを続けて書いたものである.--forceオプションは,ファイルを削除する際の確認をスキップする機能であり,--recursiveオプションは,ディレクトリを指定すると,そのディレクトリと中身を再帰的に削除する機能を持つ[2].
最後に,「/」はLinuxのルートディレクトリを指す記号であり,全てのファイルとディレクトリの親にあたる[3].

以上を踏まえると,「rm -rf /」というコマンドは,Linux内の全てのファイルとディレクトリの親であるルートディレクトリおよびその中身を,確認なしで再帰的に削除するコマンドだとわかる.
故に,「rm -rf /」を実行することは極めて危険である.
\section{感想}
私は普段からMacやUbuntuなど,LinuxやLinuxに似た環境で作業を行い,コマンドを使用している.しかし,「Permission denied」というエラーに対処する際,sudoコマンドを使用していたものの,実際には何の権限が不足しているのか,また権限にはどのような種類があるのかを深く理解していなかった.
今回の授業を通じて,具体的な対処法やその対処法が機能する仕組みについて学ぶことができた.その結果,単にsudoを使うだけでなく,権限の仕組みを理解した上で説明できるようになり,非常に有意義な学びとなった.また,LinuxとMacのコマンドがよく似ている理由や,これらのOSの関係性,さらにはそれぞれの特性の違いについても興味が湧いた.これをきっかけに,さらに理解を深めていきたいと感じている.

\begin{thebibliography}{9}
    \item Linuxコマンド集 \$ rm」, リスキルテクノロジー, \url{https://eng-entrance.com/linux_command_rm}, \today 参照.
    \item rm コマンド」, IBM Corporation, \url{https://www.ibm.com/docs/ja/aix/7.2?topic=r-rm-command}, \today 参照.
    \item 三宅英明, 大角祐介, 「新しい Linux の教科書 第 2 版」, SB クリエイティブ, pp.52-53, 2024.
\end{thebibliography}

\end{document}

